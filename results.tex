\section{Results}                                    
    For the category-based analysis we have used the predefines categories from the stimulus presentation, as described in section~\ref{sec:m_st}. 
    \subsection{Baseline Reaction Time}\label{sec:r_brt}
		Here we analyse the reaction times in trials containing equally attractive faces (attractive-attractive, or unattractive-unattractive).
		
		\py[coni]{fig_coni}
		
		Figure~\ref{fig:coni} shows that attractive-attractive and unattractive-unattractive conditions prompt the same reaction times, with a t-test \textit{p}-value of 
		\py[coni]{tex_nr(ttest_rel(data_coni[data_coni['block']=='aa']['RT'], data_coni[data_coni['block']=='uu']['RT'])[1])}. 
		Based on this we conclude that the mean reaction time for these categories is suitable as a baseline for referencing future measurements.
		
		The population mean reaction time is
		\py[coni]{int(around(mean(data_coni[(data_coni['block']=='aa') | (data_coni['block']=='uu')]['RT'])*1000))} ms.
    \subsection{Conditions of Interest}\label{sec:r_coi}
		Here we analyse the reaction times for trials where the target cue is flanking an attractive face and of trials where the target cue is flanking an unattractive face.
		
		\py[coi]{fig_coi}
		
		Figure~\ref{fig:coi} shows that there is a significant difference in reaction times between conditions where the target is flanking an attractive face and conditions where the target is flanking an unattractive face.
		The t-test \textit{p}-value for the comparison of these conditions is 
		\py[coi]{tex_nr(ttest_rel(data_coi[data_coi['subblock']=='aus+sua']['RTdiff'], data_coi[data_coi['subblock']=='uas+sau']['RTdiff'])[1])},
		and the population mean difference of reaction times is
		\py[coi]{int(around((data_coi[data_coi['subblock']=='uas+sau']['RTdiff'].mean()-data_coi[data_coi['subblock']=='aus+sua']['RTdiff'].mean())*1000))} ms.
		
		This population effect is also robust at the per-participant level.
    \subsection{Correlation Analysis}\label{sec:r_ca}
    \subsection{Eye Tracking}\label{sec:r_et}
		To examine how attractive and unattractive faces manipulate the gaze location to effect the aforementioned difference in reaction times, we analyse the eye tracking data.
		For the following figures we have calculated the means and standard errors across participants in each of the search task conditions.
		
		\py[ettx]{fig_ettx}
		\py[etty]{fig_etty}
		
		The time courses presented in figure~\ref{fig:ettx} show a high congruence of gaze paths on the X-axis for trials with both faces from the same attractiveness category.
		When face categories differed, trials with unattractive faces on the target cue side prompted a stronger gaze pull towards the cue side of the screen at the end of the time course.
		This effect was absent when attractive faces were on the target cue side.
		Additionally we note a greater standard error (and standard deviation, not shown) for gaze paths where the target cue was flaking an unattractive face and the distractor cue was flanking an attractive face
		- compared to all other conditions.
		
		For comparison purposes we show the same analysis along the Y-axis in figure~\ref{fig:etty}.
		Here we notice a longer delay until the gaze coordinates become jittered for trials where there is a difference in the attractiveness category of the faces as opposed to trials where this is not the case.
		

\section{Methods}
    \subsection{Face Attractiveness Rating}\label{sec:m_far}
	Face stimuli from a database of 50 faces of the preferred gender of the participant were selected and presented for a duration of \SI{2}{\second}; each face was presented thrice.
	Following the presentation of each face the display was blanked and a visual analogue scale (VAS) was displayed until an input was provided by the participant.
	Each trial consisting of a face and a VAS display was separated from the next by a fixation dot display lasting \SI{2}{\second}.
	
	All stimuli were presented completely randomized.
    \subsection{Search Task}\label{sec:m_st}
	From the per-participant attractiveness ranking obtained in section~\ref{sec:m_far} the top 20 and bottom 20 rated faces were automatically selected as discrete categories for the search task.
	These categories will be henceforth referred to as attractive (top 20 rated) and unattractive (bottom 20 rated).
	

	The search task displays presented 2 images each.
	Flaking the face images we presented a grey circle (the search target) and a grey diamond (search distractor).
	We presented 4 conditions of search task displays (based on the category of face stimuli and keeping in mind left/right laterality) : (I) attractive-unattractive, (II) unattractive-attractive, (III) attractive-attractive, and (IV) unattractive-unattractive.
	Each of these conditions was presented 30 times, with trial order fully randomized.
	Target sides (left/right) were counterbalanced 15-to-15 for each condition.
	
	In the search task, participants were prompted to locate the side of the display containing the search target, and signal their decision via a left or right key press.

	The search task displays were presented for a maximum of \SI{1.5}{\second}, but replaced by fixation dot displays as soon as the participant provided an input.
	Fixation intervals were constant, with a duration of \SI{2}{\second}.
